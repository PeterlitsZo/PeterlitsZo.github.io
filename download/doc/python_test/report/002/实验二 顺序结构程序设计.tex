\documentclass[a4paper]{ctexart}

\usepackage{layout}
\setlength{\oddsidemargin}{10pt}
\setlength{\textwidth}{431pt}
\setlength{\headsep}{10pt}
\setlength{\textheight}{628pt}

\usepackage{listings}
\usepackage{xcolor}
\lstset{numbers=left, 
   numberstyle= \small\ttfamily,
   basicstyle= \ttfamily,
   keywordstyle= \color{ blue!70!green!70!white},
   commentstyle= \color{red!50!green!50!blue!50}, 
   stringstyle= \color{red!70!yellow!70!blue}, 
   showstringspaces= false,
   frame= shadowbox, 
   rulesepcolor= \color{ red!20!green!20!blue!20} 
} 

\usepackage{enumitem}
\setlist{leftmargin=4em, itemsep=0.1em, parsep=0em, topsep=0.2em, 
			itemindent=0em, listparindent=0em, labelwidth=1.5em, labelsep=1em}

\newcommand{\lineiin}{
   \begin{center}
      \rule{\textwidth}{0.1pt}
   \end{center}
}

\title{苏州大学实验报告\footnote
   {注:本文档在 peterlits.com/download/doc/python\_test/report/002/index.html处}}
\author{}

\begin{document}
   \maketitle

   \begin{table}[!h]
      \centering
      \begin{tabular}{|c|c|c|c|c|c|c|c|}
         \hline
         院、系 & 计算机学院 & 年级专业 & 计算机科学 
         & 姓名 & 周泓余 & 学号 & 1927405082 \\ \hline
         课程名称 & \multicolumn{5}{c|}{Python 程序设计} & 成绩 & \\ \hline
         指导老师 & \multicolumn{2}{c|}{}& 同组实验者 & 无 & 实验日期
         & \multicolumn{2}{c|}{10月28号} \\
         \hline
      \end{tabular}
      \caption{实验报告相关信息}
   \end{table}

   \section*{实验名称:实验二~顺序结构程序设计}
   \subsection{实验目的}
      通过本次实验要达到如下目的:
      \begin{enumerate}
         \item 掌握顺序结构程序设计方法
         \item 掌握数据的输入和格式化输出方法
         \item 掌握求解问题的算法描述方法
         \item 掌握Python语言程序设计的基本规则
         \item 掌握Python语言常用模块的使用方法
         \item 熟悉从程序设计的角度考虑问题、求解问题
      \end{enumerate}

   \subsection{实验内容}
      \begin{enumerate}
         \item 从键盘输入一个3位整数,请编写程序计算三位整数的各位数字之和,%
            并输出到屏幕上,要求输出占4列,右对齐。
         \item 编写一个程序,提示用户输入三角形的三个顶点$(x_1,y_1)$、%
            $(x_2,y_2)$、$(x_3,y_3)$,然后计算三角形面积,%
            这里假定输入的三个点能构成三角形。将面积输出到屏幕,%
            要求输出占7列,保留2位小数,左对齐。\\
            三角形面积公式如下:
            $$S=\frac{side_1+side_2+side_3}{2}, area=\sqrt{S(S-side_1)(S-side_2)(S-side_3)}$$
            其中:$side_1$,$side_2$,$side_3$表示为三角形三条边的长度。
         \item 假设每月存100元到一个年利率为6\%的储蓄账户。因此,月利率为0.06/12=0.005。\\
            第一个月后,账户的存款金额为: 100*(1+0.005) = 100.5\\
            第二个月后,账户的存款金额为: (100+100.5)*(1+0.005) = 201.5025\\
            第三个月后,账户的存款金额为: (100+201.5025)*(1+0.005) = 303.3115\\
            请编写程序计算5个月后,该储蓄账户的存款金额是多少,并显示在屏幕上,%
            要求保留5位小数,右对齐。计算总体收益相对总体本金的收益率%
            (此收益率值:总收益/总本金),并显示在屏幕上,要求以百分数形式显示,保留2位小数,%
            右对齐。
         \item 请编写一个程序显示当前北京时间,要求显示格式如下:\\
            当前时间是:几时:几分:几秒\\
            输出示例:当前时间是: 14:26:32
         \item 请编写一个程序,产生一个在[5,20]之间的随机实数。假设该随机数是一个球的半径,%
            请计算该球的体积。最后将球的半径和体积输出到屏幕上,要求每个值占15列,%
            保留3位小数,右对齐。
         \item 从键盘输入两个向量,每个向量的维度是2,向量中每个元素的范围在0到1之间,%
            计算两个向量的余弦相似度,并输出结果。
         \item 从键盘输入两个时间点,格式hh:mm:ss(时:分:秒),计算两个时间点相隔的秒数,并输出。
         \item 请编写一个程序,产生两个[10,50]之间的随机数,用这两个数构造一个复数,%
            计算复数的模、辐角(要求转换成角度),最后将复数、复数的模和辐角显示在屏幕上。%
            要求每个占7列,保留2位小数,右对齐。
         \item 请计算当前距离1970年1月1日过去了多少天又多少小时,并输出到屏幕上。
      \end{enumerate}

   \subsection{实验步骤和结果}
      \subsubsection{实验之中}
      第一题程序如下:
      % <-
      \begin{lstlisting}[language=python]
   num = input('> ')
   print(sum([int(i) for i in num]))
      \end{lstlisting}
      % ->

      运行结果如下:
      % <-
      \begin{lstlisting}
   > 234
   9
      \end{lstlisting}
      % ->

      \lineiin

      第二题程序如下:
      % <-
      \begin{lstlisting}[language=python]
   def get_len_of_two_pairs(pair_1, pair_2):
      res = (pair_1[0] - pair_2[0])**2 + (pair_1[1] - pair_2[1])**2
      return res ** 0.5

   pairs = eval(input('Please Enter 3 pairs:' \
      ' such as (1, 0), (0, 1), (1, 1)\n> '))
   p1, p2, p3 = pairs

   g = get_len_of_two_pairs
   side_1, side_2, side_3 = g(p1, p2), g(p1, p3), g(p2, p3)

   S = (side_1 + side_2 + side_3) / 2
   area = (S * (S - side_1) * (S - side_2) * (S - side_3))**0.5
   print(area)
      \end{lstlisting}
      % ->

      运行结果如下:
      % <-
      \begin{lstlisting}
   Please Enter 3 pairs: such as (1, 0), (0, 1), (1, 1)
   > (0, 0), (0, 1), (1, 1)
   0.49999999999999983
      \end{lstlisting}
      % ->

      \lineiin

      第三题程序如下:
      % <-
      \begin{lstlisting}[language=python]
   def save_money(money_before):
      return (100 + money_before) * (1 + 0.005)

   money = 0
   for i in range(5):
      money = save_money(money)
   print('{:>10.5f}\n{:>9.5f}%'.format(
      money, (money - 500) / 500 * 100)
   )
      \end{lstlisting}
      % ->

      运行结果如下:
      % <-
      \begin{lstlisting}
   507.55019
    31.51004%
      \end{lstlisting}

      \lineiin

      第四题程序如下:
      % <-
      \begin{lstlisting}[language=python]
   import time
   _, _, _, hour, mine, sec, *_ = time.localtime()
   print(f'{hour}: {mine}: {sec}')
      \end{lstlisting}
      % ->

      运行结果如下:
      % <-
      \begin{lstlisting}
   23: 8: 4
      \end{lstlisting}

      \lineiin

      第五题程序如下:
      % <-
      \begin{lstlisting}[language=python]
   import random, math
   r = random.random() * 15 + 5
   V = 4 / 3 * math.pi * r**3
   print(
      "{:>15.3f}".format(r),
      "{:>15.3f}".format(V),
      sep = '\n'
   )
      \end{lstlisting}
      % ->

      运行结果如下:
      % <-
      \begin{lstlisting}
           11.807
         6894.897   
      \end{lstlisting}

      \lineiin

      第六题程序如下:
      % <-
      \begin{lstlisting}[language=python]
   p1, p2 = eval(input('> '))

   get_len = lambda p: (p[0]**2 + p[1]**2)**0.5
   res = (p1[1] * p2[1] + p1[0] * p2[0])
   res = res / (get_len(p1) * get_len(p2))
   print(res)
      \end{lstlisting}
      % ->

      运行结果如下:
      % <-
      \begin{lstlisting}
   > (1, 2), (2, 1)
   0.7999999999999998
      \end{lstlisting}

      \lineiin

      第七题程序如下:
      % <-
      \begin{lstlisting}[language=python]
   from datetime import datetime

   def get_time():
      return [int(num) for num in input('> ').split(':')]

   time_1 = datetime(1, 1, 1, *get_time())
   time_2 = datetime(1, 1, 1, *get_time())

   time_delta = abs(time_2 - time_1)
   print(time_delta.seconds)
      \end{lstlisting}
      % ->

      运行结果如下:
      % <-
      \begin{lstlisting}
   > 13:10:2
   > 12:10:23
   3579
      \end{lstlisting}

      \lineiin

      第八题程序如下:
      % <-
      \begin{lstlisting}[language=python]
   from random import random
   from math import atan2

   num_1 = random() * 40 + 10
   num_2 = random() * 40 + 10

   num_c = num_1 + num_2 * 1j
   num_c_abs = abs(num_c)
   num_c_agr = atan2(num_2, num_1)

   print(('{:>7.2f}\n'*3).format(
      num_c, num_c_abs, num_c_agr
   ))
      \end{lstlisting}
      % ->

      运行结果如下:
      % <-
      \begin{lstlisting}
   39.41+43.25j
     58.52
      0.83

      \end{lstlisting}

      \lineiin

      第九题程序如下:
      % <-
      \begin{lstlisting}[language=python]
   from datetime import datetime

   delta = datetime.today() - datetime(1970, 1, 1)
   print(delta.days, delta.seconds//3600)
      \end{lstlisting}
      % ->

      运行结果如下:
      % <-
      \begin{lstlisting}
   18218 13
      \end{lstlisting}

   \subsection{实验总结}
      通过本次实验,我学会了vs code开发环境的使用;了解了程序开发的过程,%
      加深理解程序运行的流程。学习基本操作符的使用。

\end{document}