\documentclass[a4paper]{ctexart}

\usepackage{layout}
\setlength{\oddsidemargin}{10pt}
\setlength{\textwidth}{431pt}
\setlength{\headsep}{10pt}
\setlength{\textheight}{628pt}

\usepackage{enumitem}
\setlist{leftmargin=4em, itemsep=0.1em, parsep=0em, topsep=0.2em, 
			itemindent=0em, listparindent=0em, labelwidth=1.5em, labelsep=1em}

\newcommand{\lineiin}{
   \begin{center}
      \rule{\textwidth}{0.1pt}
   \end{center}
}

\title{苏州大学实验报告\footnote{注:本文档在 peterlits.com/download/doc/python\_test/report/001/index.html处}}
\author{}

\begin{document}
   \maketitle

   \begin{table}[!h]
      \centering
      \begin{tabular}{|c|c|c|c|c|c|c|c|}
         \hline
         院、系 & 计算机学院 & 年级专业 & 计算机科学 
         & 姓名 & 周泓余 & 学号 & 1927405082 \\ \hline
         课程名称 & \multicolumn{5}{c|}{Python 程序设计} & 成绩 & \\ \hline
         指导老师 & \multicolumn{2}{c|}{}& 同组实验者 & 无 & 实验日期
         & \multicolumn{2}{c|}{10月28号} \\
         \hline
      \end{tabular}
      \caption{实验报告相关信息}
   \end{table}

   \section*{实验名称:实验一 Python 语言基础}
   \subsection{实验目的}
      通过本次实验要达到如下目的:
      \begin{enumerate}
         \item 掌握 Python 开发环境的使用
         \item 掌握变量的使用方法
         \item 了解数据的输入和输出方法
         \item 了解并学会选择数据类型
         \item 掌握算术运算符的使用
         \item 掌握math模块中常用函数的使用
      \end{enumerate}

   \subsection{实验内容}
      \begin{enumerate}
         \item 从键盘输入两个正整数 a 和 b ,计算并输出 a除以 b 的商和余数。
         \item 从键盘输入四个整数,并输出其中最大的数。
         \item 编写程序让用户输入自己姓名,输出该姓名字符串的长度。
         \item 一只大象口渴了,要喝20升水才能解渴,但现在只有一个深h厘米,%
            底面半径为r厘米的小圆桶(h和r都是整数)。问大象至少要喝多少桶水才会解渴。%
            编写程序输入半径和高度,输出需要的桶数(一定是整数)。
         \item 编写程序让用户输入两个平面上点的坐标,计算该两点间的距离。
         \item 产生一个随机3位正整数,并将该整数的数字首尾互换输出,%
            例如:157 互换后为 751。
      \end{enumerate}

   \subsection{实验步骤和结果}
      \subsubsection{实验之前}
      在给出先关代码前,先定义 \verb|print_cm| 来定义输入容错和输入输出格式。
   % <-
   \lineiin
   \begin{verbatim}
   import re


   class print_cm(object):
      def __init__(self, *, indent:int = 4, ps:str = '>>> ', newline:bool = True):
         self.ps1 = ' '*indent
         self.ps2 = self.ps1*2 + ps
         self.ps3 = self.ps1*2 + '{:2}' + ps[2:]
         self.newline = newline
         if self.newline == True:
            print()
   
      def __del__(self):
         if self.newline == True:
            print()
   
      def __format_out(self, type, addition:object or [object, '...']=''):
         if type == 'invalid format':
            self.info('you have enter in wrong format, please try a again:')
         elif type == 'in format':
            self.info('Please Enter in format like:' \
                     ' (enter only one {} one time):'.format(addition))
         elif type == 'enter':
            self.info('Please Enter {} {}:'.format(addition[0], addition[1]))
   
      def __format_re_in(self, ma_str, re_str, res_f, times = 1) \
         -> [tuple(str and '...') or 'else... (by res_f)']:
         """to input a format data
   
         ma_str(match re string): to match the target string
         re_str(re string): input the re string to get data in format str or str tuple
            : matched str -> tuple of str
         res_f(result geting function): return the result by this result_func
            : tuple('str', ...) -> the object you want
         times: the len of the result list, or the times of loop
         """
         res = []
   
         for i in range(times):
            while True:
               input_ = self.enter(info=i+1)
               # type of result: list of ‘str’ or tuple(‘str, ...)
   
               if input_ == '':
                     # if here has no input at all
                     continue
               elif re.fullmatch(ma_str, input_):
                  res_tuple = tuple(re.findall(re_str, input_))
                  res.append( res_f(res_tuple) )
                  break
               else:
                  self.__format_out('invalid format')
                  self.info(f'What you had input is: {res}')
         return res
   
      def enter(self, type:str = None, times:int = 1, *, info = None) \
         -> [object, '...']:
         re_float = r'(\d+(?:\.\d*)?|\.\d+)'
         if type == 'int':
            self.__format_out('enter', [times, 'int num'])
            return self.__format_re_in(r' *[\+|-]? *\d+ *', r'\d+', \
               lambda x: int(*x), times)
         elif type == 'name':
            self.__format_out('enter', [times, 'str'])
            self.info('Please to know about it:')
            self.info("iuput: ' Peterlits  Zo  '(ugly space) ->" \
               " output:'Peterlits Zo'", indent=2)
            result_func = lambda res: ' '.join([i for i in res])
            return self.__format_re_in(r'\s*(\w+\s*)+', r'\w+', result_func, times)
         elif type == 'pair':
            self.__format_out('enter', [times, 'pair(2-d vector)'])
            self.__format_out('in format', 'pair')
            self.info("(4, 8) or (9.0, 2) or (.4, 0.2)", indent=2)
            result_func = lambda res: tuple(float(i) for i in res)
            re_str = r'\s*\(\s*{__f__}\s*,\s*{__f__}\s*\)\s*'.format(__f__ = re_float)
            return self.__format_re_in(re_str, re_float, result_func, times)
         else:
            if info == None:
               return input(self.ps2)
            else:
               return input(self.ps3.format(str(info)))
   
      def info(self, *info:str, indent = 1, with_ps = False):
         for i in info:
            if with_ps == True:
               print(self.ps1*(indent-1) + self.ps2, i, sep='')
            else:
               print(self.ps1*indent, i, sep='')
      \end{verbatim}
      \lineiin
      % ->
         通过\verb|print_cm|对象实例的方法\verb|enter|和\verb|info|来进行输入输出。%
         其中\verb|enter|支持\verb|pair|,\verb|int|,\verb|name|格式,%
         提供人性化的可交互的输入输出。在使用该类的情况下:%
         (所有代码在\verb|python3.8.0|环境运行)

         \subsubsection{实验之中}
         第一题程序如下:
      % <-
      \begin{verbatim}
      pc = print_cm()
      pc.info('请输入数字以求取商和余数:')
      a, b = pc.enter('int', 2)
      pc.info(f'{a} 除以 {b} 的商:{a//b}', f'{a} 除以 {b} 的余数:{a%b}')
      \end{verbatim}
      % ->

         运行结果如下:
      % <-
      \begin{verbatim}
      请输入数字以求取商和余数:
      Please Enter 2 int num:
         1 > 324df
      you have enter in wrong format, please try a again:
      What you had input is: []
         1 > 23
         2 > 243tf
      you have enter in wrong format, please try a again:
      What you had input is: [23]
         2 > 12
      23 除以 12 的商:1
      23 除以 12 的余数:11
      \end{verbatim}
      % ->

         第二题程序如下:
      % <-
      \begin{verbatim}
      pc = print_cm()
      pc.info('请输入数字以求取最大数')
      a_list = pc.enter('int', 4)
      pc.info(f'最大的数为{max(a_list)}')
      \end{verbatim}
      % ->

         运行结果如下:
      % <-
      \begin{verbatim}
      请输入数字以求取最大数
      Please Enter 4 int num:
         1 > 342
         2 > 4653
         3 > dfs
      you have enter in wrong format, please try a again:
      What you had input is: [342, 4653]
         3 > 3547
         4 > 234
      最大的数为4653
      \end{verbatim}
      % ->

         第三题程序如下:
      % <-
      \begin{verbatim}
      pc = print_cm()
      pc.info('请输入姓名以求取姓名的长度:')
      a, = pc.enter('name')
      pc.info(f'格式化后输入的字符串为{repr(a)},总长度为{len(a)}')
      \end{verbatim}
      % ->

         运行结果如下:
      % <-
      \begin{verbatim}
      请输入姓名以求取姓名的长度:
      Please Enter 1 str:
      Please to know about it:
         iuput: ' Peterlits  Zo  '(ugly space) -> output:'Peterlits Zo'
         1 > Pfesjf4    reganj    
      格式化后输入的字符串为'Pfesjf4 reganj',总长度为14
      \end{verbatim}
      % ->

         第四题程序如下:
      % <-
      \begin{verbatim}
      import math
      pc = print_cm()
      pc.info('请分别输入水桶的深和底面半径,以求出一个特定的大象的喝水桶数:(cm)')
      h, r = pc.enter('int', 2)
      res = 20 * 1000 / (math.pi * r ** 2)*h
      pc.info(f'需要的桶数为{math.floor(res)+1}')     
      \end{verbatim}
      % ->
      
         运行结果如下:
      % <-
      \begin{verbatim}
      请分别输入水桶的深和底面半径,以求出一个特定的大象的喝水桶数:(cm)
      Please Enter 2 int num:
         1 > 
         1 > 
         1 > 1
         2 > 7
      需要的桶数为130
      \end{verbatim}
      % ->

         第五题程序如下:
      % <-
      \begin{verbatim}
      pc = print_cm()
      pc.info('请输入平面上的两个点:')
      p1, p2 = pc.enter('pair', 2)
      get_len = lambda p1, p2: ((p1[0]-p2[0])**2 + (p1[1]+p2[1])**2)**.5
      pc.info(f'两点之间的距离为{get_len(p1, p2)}')
      \end{verbatim}
      % ->

         运行结果如下:
      % <-
      \begin{verbatim}
      请输入平面上的两个点:
      Please Enter 2 pair(2-d vector):
      Please Enter in format like: (enter only one pair one time):
         (4, 8) or (9.0, 2) or (.4, 0.2)
         1 > (7,6)
         2 > --
      you have enter in wrong format, please try a again:
      What you had input is: [(7.0, 6.0)]
         2 > yi
      you have enter in wrong format, please try a again:
      What you had input is: [(7.0, 6.0)]
         2 > (8,0)
      两点之间的距离为6.082762530298219 
      \end{verbatim}
      % ->

         第六题程序如下:
      % <-
      \begin{verbatim}
      import random
      rand_int = random.randint(100, 999)
      pc = print_cm()
      pc.info(f'产生的随机数字为{rand_int},互换后为{str(rand_int)[::-1]}')
      \end{verbatim}
      % ->

         运行结果如下:
      % <-
      \begin{verbatim}
      产生的随机数字为191,互换后为191 
      \end{verbatim}

   \subsection{实验总结}
      通过本次实验,我学会了vs code开发环境的使用;了解了程序开发的过程,%
      加深理解程序运行的流程。学习基本操作符的使用。

\end{document}