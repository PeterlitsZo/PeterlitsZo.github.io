\documentclass[b5paper]{ctexart}

\usepackage{amsmath}
\usepackage[colorlinks,linkcolor=black]{hyperref}
\usepackage{color}
\usepackage{calc}
\usepackage{layouts, layout}
\usepackage{enumitem}

\usepackage{tikz}
\newcommand*{\circled}[1]{
	\tikz[baseline=(char.base)]{
		\node[shape=circle,draw,inner sep=1.3pt] (char) {   
			\small{#1}
		}
	}
}

\setlist{leftmargin=4em, itemsep=0.1em, parsep=0em, topsep=0.2em, 
			itemindent=0em, listparindent=0em, labelwidth=1.5em, labelsep=1em}

\newcommand{\mathline}{\_\_\_\_\_\_\_\_}
\newcommand{\linein}{
	\begin{center}
		\rule{0.5\textwidth}{0.1pt}
	\end{center}
}
\newcommand{\lineiin}{
	\begin{center}
		\rule{\textwidth}{0.1pt}
	\end{center}
}

\title{学习方法总结}
\author{\href{mailto:peterlitszo@outlook.com}{Peterlits}\footnote{在PDF格式下点击后就可以给作者发送电子邮件。纸质文档下,就麻烦在邮箱软件中键入下方的peterlits\-zo\-@\-outlook.com来给我发送邮件了。通过电子邮件给我意见或者获取这份文档的最新版,也欢迎在相关的git项目下fork、修改、或者提交请求。}\\\small{<peterlitszo@outlook.com>}}

\begin{document}
	\maketitle
	\newpage
	\tableofcontents
	\newpage

	\thispagestyle{empty}
	\parbox[t][15em]{\textwidth}{}

	本篇文章禁止用于商业用途,所有素材来自与专业相关的书籍、生活中的实践、朋友以及网络,在此谢谢bilibili的up主蜡笔和小勋(虽然不一定会看见)。
	本文档使用B5纸。

	\ \newline

	献给我的朋友 --- 冉一丹,希望高考顺利。

	\ \newline

	\section{时间规划}
		\begin{quote}
			孩子,发现自己眼中的光。

			然后站起来,向着那光寻去。

			\begin{flushright}
				--- 舟彼
			\end{flushright}
		\end{quote}

		在忙碌的学习生活中,时间总是显得不够用。在时间规划上要注意\emph{长期目标高估自己,短期目标低估自己}这一原理。五天改变不了什么,五年能改变很多。

		所以,你要学会怎么正确地压榨自己\footnote{注意参考第\ref{Sec::身体}节。}。

		以下由短到长讲讲业界公认的对学习有帮助的办法:

		\subsection{番茄表}
			简单来说,番茄钟是一个由25分钟和5分钟组成的时间规划系统,从进行、打断等各方面都提出了自己的建议。

			但是有时番茄法也有不好的地方:它比较容易和学校的上课、自习时间安排冲突。这种时最好就老老实实地按照学校的课程表来进行作息调整。毕竟相应的课程时间安排也是考虑了很多方面而尽量做出的最优解。

			在学校时间安排之外,比如,在假期这种容易分神、不走心的情况下,使用番茄一般会取到意想不到的长效作战时长\footnote{比如我现在正在使用番茄钟、从早上的10点钟做到了晚上8点。它长效的最主要原因是会让你强制休息,所以身体和精神都不会过度疲惫,还保证了效率。}。

			所以说,什么是番茄钟呢?又该怎么使用番茄钟呢?

			\href{https://zh.wikipedia.org/zh-hans/%E7%95%AA%E8%8C%84%E5%B7%A5%E4%BD%9C%E6%B3%95}{维基}\footnote{我现在也进不去,目前好像需要翻墙。}%
			中,定义了番茄钟的基本使用方法和详细的使用见解,有条件可以去看看。

			番茄钟,主要是从以下五个方面起作用而来提高自己学习效率的:
			
			\begin{enumerate}
				\item \emph{解决走神},番茄中的核心计时器是用来分割时间块的,计时器跟踪工作会带来时间上的紧迫感,所以就不会过多地走神(\emph{比如,``我还剩最后5分钟了,没有多少时间了\ldots\ldots ''之类的})。
				\item \emph{使自己着眼于当下},不会过多的考虑之后该做什么,和当前应该做什么,也不会在中途时改变计划,因为在番茄钟开始的时候就已经有相应的计划,所以一般能至少完整地走完一个番茄钟。
				\item \emph{手机被占用},一般在没有外部者管理的情况下,手机很容易造成分心。转换手机的定位是一个不错的方法\footnote{当然把手机放在\emph{低危区}也是一个非常不错的办法。},把手机从一个娱乐工具变成一个督促者的身份会大大降低使用手机娱乐的欲望。
				\item \emph{反射的力量},除此之外,二十五分钟、不可分割,不可中断这些条件都让大脑一步一步加深影响,久而久之就形成了一个条件反射。有人说过,坚持很难。可以,如果已经形成习惯了,那阻力难道还会那么大吗?
				\item \emph{奖励的快感},每个番茄钟都有五分钟的休息。一个小奖励,不仅会让自己开心\footnote{比如说:哇!这是我辛辛苦苦工作了二十五分钟换来的!太棒了!},也可以让自己休息。不仅没什么累的,成就感也很高,效率也有了相应的保障。
			\end{enumerate}

			那么番茄钟是如何工作的呢?
			\begin{itemize}
				\item 25分钟 --- 只能学习,不能做其他任何事情。
				\item 5分钟 --- 马上放下笔,强制休息5分钟。
				\item 每四个番茄钟(\emph{就是说,四个由25分钟和5分钟构成的循环})后有三十分钟的长休息(\emph{对应5分钟的短休息}),合理安排长休息的时间,在放松自己的同时可可以干不少事情。 
			\end{itemize}

			在番茄钟的休息时间中,需要注意的两点是:一是要离开座位、适量的刺激身体;二是尽量不用脑。B站up主\emph{蜡笔和小勋}\footnote{是一对十分优质的、神仙学习区视频主。}说过,在番茄钟期间,要``动动胳膊动动腿,上个厕所喝口水'',如是也。

			总之,休息时不用脑;用脑时就一定正在工作或学习的时候。

			番茄钟掌管小方向,自然会难以顾全大局。使用日计划表来规划出今日的计划清单,让自己努力的每一分钟都对得起自己所用的时间。

			25分钟说短但也不短,就可能会因为一些小事或突发事件而造成打断。处理学习过程中的打断,主要是从两点分析,一是\emph{自我打断};二是\emph{外部打断}。%
			处理这两个打断,因为产生的原因不同,所以解决方法也当然不尽相同:
			\begin{itemize}
				\item 自我打断:
				\begin{itemize}[leftmargin=2em]
					\item \emph{原因}:自我打断,就是自己打断自己。人是一种很容易受外界影响的生物,也容易受自己影响的生物。所以在番茄钟的工作周期里时总会想到其它的事情,也就是俗称的开小差。
					\item \emph{方法}:首先要确认的是,如果不是特别紧急的话,是一定不能终止番茄钟的。其次,出于白熊效应\footnote{白熊效应即,如果强迫人脑不去想某件事,那么大脑会反弹得更多地想那件事。比如实验员曾经让实验者不要想白熊和和白熊有关的事,但是自顾自的努力却往往没用,反而还会反弹得更多。},首要想的不是抗拒去想这件事,需要的是接受它。然后在这个时候最好应该快速地把它记在备忘录上并暗示自己番茄钟完了之后就会立即着手去做、然后、马上回到正常的学习中。
				\end{itemize}
				\item 外部打断:
				\begin{itemize}[leftmargin=2em]
					\item \emph{原因}:外部打断发生在外部环境中。
					\item \emph{方法}:解决外部打断分三步走:告知$\to$承诺$\to$记录。首先告诉打断自己的人:自己正在学习中,之后可能才有空/才可以弥补这件事;然后许下承诺:在$\times\times$时候可以完成许下的承诺;最后记录:把相关的计划记在日计划表上、或者月计划表上。
				\end{itemize}
			\end{itemize} 
			
			通过番茄工作法,及时调整自己的定位、优化日计划表、使计划更加符合自己的个人生活习惯。必要时还可以建立一张跟踪表格\footnote{个人认为平常的周末就不是很必要了,在假期使用会有一定的效果。当然看自己的情况来做决定是最好的。},这样可以更加明白自己的学习情况。

			另外有一件事情可能需要注意:最好的休息是做不同的工作。尽量尝试把不同性质的工作交替进行。
		
		\subsection{日计划表}
			日计划表其实代表了两种不同的时间规划表,但是又互相相辅相成。%

			在平时有规律的生活中(\emph{比如说每一周的学校时间安排都一样,每个星期的时间安排都一样,假期也是每天都一样{\color[gray]{0.5}无所事事}}),相对于没有规律的日子里,想要干好一件事情总是会显得更加有效率、也更有方法\footnote{同时最好记住,一天是决定不了什么的,但五年就可以决定很多。}。这种学习效率是没有规律的日子中的学习效率所不能及的。

			对于学习时间来说,了解自己的需求
			\footnote{比如说:如果我经常有不会的题,经常有搞不清的知识点,那么我每天应该留点时间问问老师;如果我没有复习这一天学习情况的习惯,那么我应该每天固定一个时间段来小小的复习;如果我吃饭后经常没有事干,那么我会想要规划好这个时间段,比如说是用来洗衣服还是\ldots\ldots{};如果我可能有意外的、计划之外的情况发生,那么我想我需要空出适量的时间来看到时候怎么安排。}
			首先找准大方向的日计划表(\emph{之后称为日固定分配时长表}),之后每天又新建一份当天日计划表(\emph{之后称为今日计划表}),根据周计划表和固定的日分配时长表,当然还要根据根据老师布置的作业、同学的请求,每天动态分配时间、得到有张有弛的今日计划表。
			(\emph{比如按照今天的作业和分配日计划表的\emph{做作业时长}来细分做不同作业的作业时长})。

			如果时间块足够大的话,可以使用番茄工作法来分配时长 ----- 这在假期或者是周末的话可以保持良好、有效率且高强度的主动学习时长。但是方法是给人用的,所以在注意休息和保证学习效率的情况下,也不能放过一些碎片时间。总之,日计划表链接了宏观计划和微观计划,持续做今日计划表和优化相对固定日分配时长表,则会让宏观计划从蓝图变成现实操控、可触摸的工作和计划。

		\subsection{月计划表}
			月计划表是一种相对比较宏观的表格。指定月计划表之前,你需要知道,到底有什么事情是你想要做的 ----- 那些相对显得\emph{重要不紧急}的计划。

			小一点的,比如想学个小吉他。大一点,比如想考上受分有点点小高的清华北大。像这种很重要(\emph{至少比今天下午班主任要在全班开会更重要一点})但是一时半会又急不来的事情,这种必须要努力实现但是又需要较长时间的计划,就需要使用目标表和月计划表了。

			首先,需要知道你的目标(\emph{制定目标表}),然后要进行\emph{目标量化}\footnote{我想要考上清华,所以我语文应该\ldots\ldots{}分,数学应该\ldots\ldots{},然后这个月我应该要在这门学科上考到\ldots\ldots{}分,需要刷题\ldots\ldots{},要背知识点\ldots\ldots{}页。}后用月计划表来安排每天的计划,最后落实在日计划表上。

			有时也有一些有时间跨度的任务,把任务记在相应日期的表格中就不大需要备忘录和deadline了。有一些重复性高的小任务\footnote{不是可以记在日分配时长上的每日重复 --- 吃饭、睡觉什么的,而是隔一天洗一次衣服、每天之类追踪、管理背书之类的的重复性工作。},有效利用好月计划表可以保证每天的工作井然有序。

			使用月计划表和遗忘曲线规律(\emph{可以了解一下351-351学习法})可以有效、牢固地记住很多知识。

		\subsection{目标表}
			这是你的最大的宏观计划表。可以试着定高点,谁还没有什么梦想呢?

	\section{学习方法}
		\begin{quote}
			实际上妈妈是对的:坦克是易朽的,而梨子是永恒的。
			\begin{flushright}
				--- 米兰$\cdot$ 昆德拉
			\end{flushright}
		\end{quote}

		\subsection{笔记 --- 康奈尔笔记法}
			如何做好一份真正利于学习的笔记?或者说,什么笔记才是利于学习的呢?

			答案是:一份强调提炼总结的笔记方法。单纯地记下老师讲的内容是难以加以记忆的,在老师的讲课基础上提炼总结,得到大纲 ------ 用这种方法是可以有效地记下知识的大体结构的。同时需要注意的是,尽量去记住而不是背住,背东西效率一般不好,而且效果也总是不尽人意。

			在笔记本上的左侧距离边距 0.3 页面宽度处画一根竖线,在页面下方画上一个横线。这样,一个\emph{康奈尔笔记法}的大体就出现了。上课时在正文区粗略地记下老师所讲的内容,下课在左侧归纳处知识结构,在下方写下总结。

			------ 而这已经被证明为一种简单有效的方法。

			在同文档目录下的\href{index.html}{index文件}处有一些康奈尔笔记的模板的链接。

			尝试打印一些以用来结构化自己的笔记。

		\subsection{分区 --- 高危区、中危区和低危区}
			出人意料的是,我们的生物本能会鼓励我们走神。走神的时候很容易获得小小的成就感,而这也让大脑释放出更多的多巴胺,形成了条件反射后一到作业面前就难免会有恶性循环。

			减少走神最有效的办法是建立一个有效的手边管理区,提高走神的成本,把走神遏制在萌芽中。

			把做作业是能手够到的称为\emph{高危区},把做作业时需要站起来的区域称为\emph{中危区},把做作业时站起来还要走几步路的称为\emph{低危区}。高危区内只能放相关的学习用品,中危区放一些不大可能用得到的学习用品,而低危区就放自己的容易诱惑自己思想的事物。

			自控力并不是在诱惑自己思想的事物构成的环境中学习而能锻炼出来的。注意!\emph{并不能}!所以在学习中适当的建立一份没有过多危险的手边管理器,形成做事利索的习惯后自然自控力也不会低。

			注意,每个人每天的自控力是有限度的,不要过多地消耗自己的自控力。

		\subsection{重复 --- 351-351学习法}
			根据遗忘曲线,人是很难在一时间能就完全掌握需要理解和背诵的知识的。

			这时候,建立一个长期的、注意重复的复习计划,就会显得很重要。351-351学习法就是一个结合了遗忘曲线而来建立起来的学习方法,首先强调了重复,其次强调了在有效记忆的效果下最少的重复取得最好的效果。

			首先它指出,在一天中学得新知识后面的第3个、第5个、第10个小时和第3天、第5天和第10天都应该有意识地记忆、重复\footnote{虽然它是这么说的,但是我一般只会在第二天、第五天和第十天才会复习}。

		\subsection{错题 --- 错题本}
			错题本在学习过程中一直有着重要的地位。

			首先错题本虽然叫错题本,但是不应该只有错题。错题本的目标是找到不会的题,所以说一个做对的题有不会的选项,一个会做的题但却总是很迷糊,这些题都是需要记在错题本上的(有时候甚至这种题会比做错的题还多)。

			其次,错题本重要的不是错题,而是思路和知识点(而这两个就是考试要考的仅有的东西了)。

			有人提议整理错题的时候使用剪贴是一个很省时间的办法,(当然我没有这个习惯,因为我会省着抄,而且图画得也将就,写错题对我来说并不是一个费时间的事情,相反它对我来说可能还有点用。

			注意,反思和总结。在学习的过程中很容易出现``\emph{虚假的努力}\footnote{就是说,只动手,不动脑,其实整整齐齐的抄下错题一般来说是对学习一点用也没有的(笑,那看来我也是虚假的努力了),动手做出成果很容易出成就感,但动脑好像就没那么容易了,为了更好的学习效果,多动脑,少动手。总之,动手时一定要边动手边动脑。拒绝虚假努力从我坐起。}''

		\subsection{结构 --- 整理知识结构}
			整理知识结构,就是一个由点串面的过程。在这个基础上被证明有效的\footnote{一定要选对自己有用的方法,做一个由目标驱动的人而不是由任务驱动的人 --- 也就是说,要明白自己到底要干什么。要做的事情不是要变得很努力,这是表象,你努力的目的才是你愿意用自己的时间去做的事情。从小处说起,就是要考个好成绩,从大处说起 --- 其实人也都会迷茫,人生就只要一次,要明白什么才会真正地让自己开心。}方法,大体上分为两类,于是就打算从两类中分别拿两种出来讲讲。

			一,动作上。尝试自己给自己讲解知识结构。研究表明,被动的听、主动的思考、和主动地教授相关知识点,它们对于实行者的学习效率的影响是从低效到高效依次递加的。其实无论是什么学习方法,最重要的是实行者必须要主动,必须有思考的过程。杜绝虚假的努力。

			尝试自己和自己讲解相关的知识点,讲完之后再看看是否完整。自己讲解给自己听,可以有效地提高吸收知识的效率。

			二,书面上。整理知识结构,就必须要对知识的数据结构熟悉并且掌握。一般来说的话,使用简单的思维导图就能取得很好的效果。

			一定要多想想``\emph{以目标为导向}''这句话,在你东寻西觅到处找知识点后做出来的思维导图肯定是对你有帮助的(\emph{而且一般帮助还不小})。但是我们做思维导图不是为了到处去找知识点的,而是去尝试区复盘整个知识结构的一个过程。

			所以说,做思维导图之前一定要自己先动脑想想知识点到底有什么。

			比如说物理中最简单的运动的描述,它涉及了位置、时间、位移、速度、加速度五个大类。其中位置中有``参考系''的知识点,而加速度有``加速度与速度之间的关系''、``加速度的方向性'',``平面上的加速度(直线和非直线)''、``加速度和力''等等等等。
		
			整理知识导图真的很费时间。

			但是我开始学物理的时候理不清很多物理概念之间的关系的时候,是从第一章的思维导图做到最后一章,成绩才慢慢有了起色(\emph{虽然之后就是完全靠错题本来``由点及面''来让物理概念之间相互有联系的})。我想这应该也会对你也有或多或少的帮助的。

		\subsection{意识 --- 三秒禅}
			(\emph{这个名字很奇怪,我自己一般没怎么使用过\ldots\ldots,当然用还是用过的})

			这是一个终止拖延的好办法。因为人的大脑一般会关注两个方面,一个是短期,另外一个是长期。在拖延的时候,(\emph{比如玩手机})大脑不幸陷入了点击$\to$获得成就感$\to$点击$\to$获得成就感的循环之中。目光就只聚集在短期的快乐中(\emph{后来放下手机就会发现快乐也并不是很快乐,就会有点点觉得自己什么都没有做,感到有点空虚})

			三秒钟,足够把自己从小循环中脱离出来,把眼光从短期中转移到长期计划中。说实话,足够了,自己的大脑会知道当下自己最想要的是什么。

			当自己及时发现自己的拖延行为时,及时止损,给自己三秒钟:
			\begin{enumerate}
				\item 现在在干什么?
				\item 自己应该干什么?
				\item 自己到底想要什么?
				\item 干了之后会有什么结果?
			\end{enumerate}

			其实自己发现自己在拖延就很困难了\footnote{因为我自己就是一个拖延症患者,自己定的计划很难在那一天全部做完}。如果真的这样的话,可以试一试天天冥想,可能会有效果。

		\subsection{复习 --- 滚动复习}
			在掌握一个知识之前,可以尝试复习两遍。

			首先先慢慢的复习一遍,尝试理解或者掌握住所有的知识点,然后快速过一遍,而这一遍,主要是对照自己的脑内知识储备和书本上的知识。简单来说就是一个加深印象和查漏补缺的一个阶段。

		\subsection{专注 --- 上课}
			可能也是这本书里最重要的一节。

			上课好好听讲,下课好好做作业,保证自己和老师想地保持在一个频率就好了。认真学可能也不会比到处使用方法的效率低。

			上课占了大部分学习时间,放弃上课实在是得不偿失的一个行为。

			最后我想说,使用上述的学习方法一般来说都可以提高效率,但是,在提高效率的同时,也要问自己到底需不需要这些所谓的``学习方法''、到底需要什么``学习方法''、自己到底喜不喜欢现在的``学习方法'',还有,永远记住 --- 适合自己的才是最好的。

	\section{身体}\label{Sec::身体}
		\begin{quote}
			圣人见微已知萌,见端已知末。
			\begin{flushright}
				--- 韩非子
			\end{flushright}
		\end{quote}
		\subsection{锻炼}
			我认为,在学校阶段没有比跑步更加有效率,又更加有意思的运动了。

			适量的运动,对于学习生活而言,最重要的作用不在于强身健体(\emph{当然这也很重要}),它最重要的在于它可以保持精神状态的稳定。

			研究表明,每天十五分钟的散步可以有效预防人们抑郁、帮助人们缓解压力,提高和改善第二天的学习状态。此外还锻炼了身体,加强了免疫力。

			每天晚上在操场跑到出汗、然后回家或者回寝室洗澡,也是给一天生活画上句号的一个特别棒的方式呢。
		\subsection{睡眠}
			在高强度的学习环境下,保证睡眠充足是非常重要的。

			至少每天都应该睡七八个小时。如果对睡眠时不管不顾的话,任自己在上课或者学习时间昏昏欲睡,是一定一定得不偿失的。

			为了获得良好的睡眠,有以下几点方法:
			\begin{enumerate}
				\item 在睡觉前可以试着做做运动(\emph{一定要提前做、因为做完运动之后会有一定时间会亢奋})、洗洗澡、泡泡脚。\\
					(\emph{让自己的体外温度和体内温度的温度差较大,会让大脑提醒身体去睡觉。而上面的方法都是提高身体温度后、又通过蒸发来减低体外温度})
				\item 睡觉保持身体不动,脑袋放空。胡思乱想得太久会让自己没有睡意。
				\item 在中午、或者小课间的时候补补觉。无论怎样,精神状态不好就很难保持一个好状态去学习或者听课。
				\item 要拎得清休息时间和学习时间。休息时间去过度学习只是给自己一个心理安慰而已(当然,在睡前用十分钟来复习。
			\end{enumerate}
		\subsection{冥想}
			(现在我并没有实践过冥想,这些都是相关资料的整理)18世纪的英国作家、批评家Samuel Johnson说过:
			\begin{quote}
				满意来自内心,那些对人性一无所知的人总是妄图改变外在而不是内在的性情追求幸福,结果总是徒劳无功,而本来想摆脱的痛苦却与日俱增。
			\end{quote}

			想要真正改变生活的话,就应该要察觉当下的位置\footnote{可是这么说的话,在平日里多思考,多批评自己,其实也是可以达到效果的,但是冥想的话其实没有特别的那么宗教化。只是很多宗教都很依赖冥想,又说一些玄而又玄的东西。}。

			冥想的本质就是\underline{主动的}~\!接纳你的想法和感受而不是特意去排斥\footnote{这么说的话,其实我认为写日记也是一个很好很好的习惯哦。}。冥想,就是让我们察觉到自己是如何产生自己的想法,而不是被想法牵着鼻子走。

			冥想原本是古老的东方修行方法\footnote{冥想真的具有很浓厚的宗教气息 --- 至少在我们所处的文化环境中。},但这现在已经不重要了,下一步是开始。

			开始冥想只需要简单的三步:
			\begin{enumerate}
				\item 坐好,后背挺直,然后闭上眼睛。
				\item 注意自己呼吸的频率。(呼气\~{}\~{}吸气\~{}\~{})选择一个感受突出的地方\footnote{通常是你的鼻部、胸部或者是腹部。},然后集中注意力,关注自己呼吸时的感觉(呼气\~{}\~{}吸气\~{}\~{})。
				\item 一开始会开始疯想,乱七八糟地想。这并不重要,重要的是你要注意你是什么时候开始走神的。然后重复。
			\end{enumerate}

			最后你会开始切实地关注当下正在发生的事。而整个冥想的过程就是尝试、失败、重新再来、失败、重新再来。
			
			你应该每天都花10钟左右来做冥想。

		\section{}
    
\end{document}