\documentclass{ctexart}

\begin{document}
	\paragraph{footnote-1}
	模型,是一种表示现实世界的抽象形式的一种表示方法,比如说,飞机模型就是从现实世界中抽象出来(不同于真正的飞机)但又与飞机相似(具有飞机的一些属性)。\\
	理想模型,就突出了被研究物体的主要属性(如物体整体运动的方面),忽略了次要方面,抽象出了只有理想情况下存在,或者根本不能存在的物体。\\
	(听说有个科学家为了帮助农民研究他母鸡不下蛋的问题,构建了一个\emph{真空环境下的球形母鸡}这一理想模型)\\
	理想化模型源于具体事物又高于具体事物。从研究物体的运动从而抽象出的质点模型(其具有一切运动物体的共性)的建立过程是一种典范,又具有普遍意义。

	\paragraph{质点模型的例子}
	老鹰在飞行时翅膀会拍打、地球在围绕太阳转时也会自转。

	\paragraph{footnote-3}
	一个物体能否看做质点由问题的性质决定。当研究问题时不需要考虑物体的形状大小时就可以将其看做一个质点。

	\paragraph{2.1.1-3}
	D

	\paragraph{时间和位移}
		\subparagraph{时间和时刻}
			使用时间坐标来表示区分时刻和时间。\\
			在坐标上,每一点反应的时间就是时刻,两点之间的间隔就是时间间隔。
		\subparagraph{路程和位移}
			位移是有方向和大小的(即,矢量),位移只有大小没有方向,是标量。(矢量的运算法则有和一般的标量不同)
\end{document}